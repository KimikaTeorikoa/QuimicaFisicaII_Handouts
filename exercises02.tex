\documentclass{tufte-handout}

%\geometry{showframe}% for debugging purposes -- displays the margins

\usepackage[spanish, es-tabla]{babel}
\usepackage{amsmath}

% Set up the images/graphics package
\usepackage{graphicx}
\setkeys{Gin}{width=\linewidth,totalheight=\textheight,keepaspectratio}
\graphicspath{{graphics/}}

\title[Química Física II: Ejercicios 2$^{o}$ Set]{
Química Física II: Ejercicios 2$^{o}$ Parcial}
%\author{David De Sancho}
\date{}  % if the \date{} command is left out, the current date will be used

% The following package makes prettier tables.  We're all about the bling!
\usepackage{booktabs}

% The units package provides nice, non-stacked fractions and better spacing
% for units.
\usepackage{units}

% The fancyvrb package lets us customize the formatting of verbatim
% environments.  We use a slightly smaller font.
\usepackage{fancyvrb}
\fvset{fontsize=\normalsize}

% Small sections of multiple columns
\usepackage{multicol}

% Provides paragraphs of dummy text
\usepackage{lipsum}

% These commands are used to pretty-print LaTeX commands
\newcommand{\doccmd}[1]{\texttt{\textbackslash#1}}% command name -- adds backslash automatically
\newcommand{\docopt}[1]{\ensuremath{\langle}\textrm{\textit{#1}}\ensuremath{\rangle}}% optional command argument
\newcommand{\docarg}[1]{\textrm{\textit{#1}}}% (required) command argument
\newenvironment{docspec}{\begin{quote}\noindent}{\end{quote}}% command specification environment
\newcommand{\docenv}[1]{\textsf{#1}}% environment name
\newcommand{\docpkg}[1]{\texttt{#1}}% package name
\newcommand{\doccls}[1]{\texttt{#1}}% document class name
\newcommand{\docclsopt}[1]{\texttt{#1}}% document class option name

\begin{document}
\maketitle
\large
\begin{enumerate}
    \item Calcula la probabilidad de que un electrón descrito por la
    función de onda 1s del átomo de hidrógeno se encuentre a una distancia del 
    núcleo igual o inferior al primer radio de Bohr.
    
    \item Calcula el radio más probable al que se encontrará un
    electrón cuando ocupe un orbital 1s de un átomo 
    hidrogenoide con número atómico Z.
    
    \item Calcula la posición exacta de los nodos de las funciones
    radiales de los orbitales 1s, 2s, 2p y 3s del átomo de hidrógeno.
    
    \item Una partícula se encuentra sometida al siguiente potencial
    \begin{equation*}
\hat{V}(x)=
\begin{cases}
  \infty & \text{ si } L<x<0\\
  0  & \text{ si } 0\leq x\leq \frac{L}{4}\text{  o  }\frac{3L}{4}\leq x\leq L
\\
  V_0, & \text{si } \frac{L}{4}<x<\frac{3L}{4}
\end{cases}
\end{equation*}
    donde $V_0$ es una constante pequeña. Usa la teoría de perturbaciones
    para calcular las correcciones de primer orden de la energía, utilizando
    como hamiltoniano de orden cero el correspondiente a la partícula en 
    una caja.
    
    \item Calcula la corrección de primer orden a la energía del 
    estado fundamental de un oscilador armónico cuya energía potencial 
    es 
    \begin{equation*}
        V(x)=\frac{1}{2}kx^2 + \frac{1}{6}\gamma x^3  + \frac{1}{24}bx^4
    \end{equation*}
\end{enumerate}

\newpage
\subsection{\textbf{Soluciones:}}
\begin{enumerate}
\item $P(r<=a_0)=0.323$
\item $r=a_0/Z$
\item 1s: No tiene nodos; 2s: $r=2a_0$; 2p: No tiene nodos (excepto $r=0$); 3s: nodos en $1.9a_0$ y $7.1a_0$.
\item $E_n^{(1)}=\frac{V_0}{2}  - \frac{V_0}{2n\pi}(\sin(3n\pi/2) - \sin(n\pi/2))$
\item $E_{v=0}^{(1)}= \frac{b\hbar^2}{32km}$; $E=E_{v=0}^{(0)}+E_{v=0}^{(1)}= 1/2\hbar\omega +  \frac{b\hbar^2}{32km}$.
\end{enumerate}
\end{document}