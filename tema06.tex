\documentclass{tufte-handout}

%\geometry{showframe}% for debugging purposes -- displays the margins

\usepackage[spanish, es-tabla]{babel}
\usepackage{amsmath}
\newtheorem{theorem}{Postulado}

% Set up the images/graphics package
\usepackage{graphicx}
\setkeys{Gin}{width=\linewidth,totalheight=\textheight,keepaspectratio}
\graphicspath{{graphics/}}

\title[Química Física II: Tema 6 - Métodos Aproximados]{
Tema 6: Métodos Aproximados}
%\author{David De Sancho}
\date{}  % if the \date{} command is left out, the current date will be used

% The following package makes prettier tables.  We're all about the bling!
\usepackage{booktabs}

% The units package provides nice, non-stacked fractions and better spacing
% for units.
\usepackage{units}

% The fancyvrb package lets us customize the formatting of verbatim
% environments.  We use a slightly smaller font.
\usepackage{fancyvrb}
\fvset{fontsize=\normalsize}

% Small sections of multiple columns
\usepackage{multicol}

% Provides paragraphs of dummy text
\usepackage{lipsum}

% These commands are used to pretty-print LaTeX commands
\newcommand{\doccmd}[1]{\texttt{\textbackslash#1}}% command name -- adds backslash automatically
\newcommand{\docopt}[1]{\ensuremath{\langle}\textrm{\textit{#1}}\ensuremath{\rangle}}% optional command argument
\newcommand{\docarg}[1]{\textrm{\textit{#1}}}% (required) command argument
\newenvironment{docspec}{\begin{quote}\noindent}{\end{quote}}% command specification environment
\newcommand{\docenv}[1]{\textsf{#1}}% environment name
\newcommand{\docpkg}[1]{\texttt{#1}}% package name
\newcommand{\doccls}[1]{\texttt{#1}}% document class name
\newcommand{\docclsopt}[1]{\texttt{#1}}% document class option name

\begin{document}

\maketitle% this prints the handout title, author, and date

\begin{abstract}
\noindent Como comentamos en el Tema anterior, la ecuación de 
Schrödinger sólo se puede resolver de manera exacta 
para átomos mono-electrónicos. Cuando un sistema tiene 
dos o más electrones es necesario recurrir
a métodos aproximados, que son principalmente dos: el
método variacional y los métodos perturbativos. 

\end{abstract}

%\printclassoptions

\section{El teorema variacional}
Sea $\hat{H}$ el operador de Hamilton y $\psi$ una función 
de onda arbitraria que utilizamos para definir un sistema, 
el \textbf{teorema variacional} nos dice que el valor 
esperado de la energía al aplicar el operador
sobre la función es siempre superior a la verdadera 
energía del estado fundamental ($E_0$). Matemáticamente, 
podemos escribir este teorema de la siguiente forma:
\begin{equation}
    E=\frac{\int{\psi^\star\hat{H}\psi d\tau}}{\int{\psi^\star \psi d\tau}} \geq E_0
    \label{eq:evar}
\end{equation}
Al cociente de la Ecuación \ref{eq:evar} lo llamamos
\textbf{integral variacional}, donde sencillamente estamos 
calculando el valor esperado de la energía para una función de
onda arbitraria, la \textbf{función de prueba}. Como no 
asumimos de partida que esta función esté normalizada,
incluimos en el denominador el término 
${\int{\psi^\star \psi d\tau}}$. 

El principio establece
que si modificamos esta función de tal manera que minimicemos
la energía, entonces nos acercaremos al máximo a la función 
de onda real. La igualdad en la Ecuación \ref{eq:evar} se 
cumple sólo en el caso en el que la función de onda de prueba 
sea idéntica a la función de onda real. El teorema variacional
proporciona una manera objetiva de aproximar la verdadera función
de onda. Este principio es la base de todos los métodos modernos
de  cálculo de estructura electrónica.

Supongamos que partimos de las funciones $\{\psi_k\}$
que son funciones propias del hamiltoniano , de tal manera que
a cada una de las funciones de onda corresponde un valor propio
para la energía,
$\hat{H}\psi_k=E_k\psi_k$. Supongamos asimismo que este conjunto
de funciones de onda forma un conjunto ortonormal. Así, la función
de onda para el estado del sistema se podrá expresar como
combinación lineal de las funciones $\psi_k$
\begin{equation}
    \psi=\sum_kc_k\psi_k
\end{equation}
y la energía correspondiente a ese estado será
\begin{equation}\begin{split}
    E=\frac{\int{\psi^\star\hat{H}\psi d\tau}}{\int{\psi^\star \psi d\tau}}=  
    \frac{\sum_i\sum_jc_i^\star c_j\int{\psi_i^\star\hat{H}\psi_j d\tau}}{\sum_i\sum_jc_i^\star c_j\int{\psi_i^\star\psi_j d\tau}}= \\
 = \frac{\sum_i\sum_jc_i^\star c_jE_{j}\delta_{ij}}{\sum_i\sum_jc_i^\star c_j\delta_{ij}}=
    \frac{\sum_ic_i^\star c_iE_{i}}{\sum_ic_i^\star c_i}
\end{split}
\end{equation}
A partir del teorema variacional podemos utilizar dos tipos
de técnicas aproximadas: el método variacional simple y el 
método de variaciones lineal.

\subsection{El método variacional simple}
En este método construimos una función de onda de prueba con 
una serie de parámetros libres ($\lambda$) y minimizamos la
energía $E$ con respecto a estos parámetros. Los pasos a dar
para identificar el valor de $\lambda$ que mejor aproxima
la función real son los siguientes:
\begin{enumerate}
    \item Definir la función de prueba.
    \item Resolver las integrales en el denominador y el numerador de la Ecuación \ref{eq:evar}.
    \item Calcular la energía aproximada como cociente del
    numerador y el denominador.
    \item Derivar la energía, $E$, con respecto a $\lambda$.
    \begin{equation}
        \frac{dE}{d\lambda} = 0  
    \end{equation}
    \item Obtener el valor de $\lambda_\mathrm{min}$ y calcular la energía para la mejor aproximación al
    estado fundamental.
\end{enumerate}

%Usaremos como ejemplo el caso del oscilador armónico, que 
%hemos resuelto de manera exacta en el Tema 3. Para describir
%el estado fundamental de este sistema, con hamiltoniano
%\begin{equation}
%    \hat{H}=-\frac{\hbar^2}{2m}\frac{\partial^2}{\partial x^2} + 
%    \frac{1}{2}kx^2
%\end{equation}
%podemos usar como función de prueba la función
%\begin{equation}
% \phi=\cos\lambda x   
%\end{equation}
%con $-\frac{\pi}{2\lambda}<x<\frac{\pi}{2\lambda}$. En
%este ejemplo, $\lambda$ es el parámetro variacional. 
%Dada la definición de la integral variacional (Ecuación 
%\ref{eq:evar}), los pasos para encontrar los valores de 
%$\lambda$ que minimizan la energía serán:
%\begin{enumerate}
%    \item Resolver la integral del denominador y el numerador:
%    operamos por separado la parte correspondiente al numerador 
%    y la energía potencial ($\hat{V}=1/2k\hat{x}^2$) y cinética 
%    ($\hat{K}$).
%    \begin{align}
%    & \int{\phi^\star\phi d\tau } = \frac{\pi}{2\lambda} \\
%    & \int{\phi^\star\hat{x}^2\phi d\tau} =  \frac{\pi^3}{24\lambda^3}-\frac{\pi}{4\lambda^3} \\
%    & \int{\phi^\star\hat{K}\phi d\tau}
%    =\frac{\lambda^2\hbar^2}{2m}\int{\phi^\star\phi d\tau}
%    \end{align}
%    Por tanto, la energía en función del parámetro $\lambda$ es
%    \begin{equation}
%        E(\lambda) = \frac{\lambda^2\hbar^2}{2m} + \frac{k}{\lambda^2}\bigg(\frac{\pi^2}{24}-\frac{1}{4}\bigg)
%    \end{equation}
%
%    \item Minimizar $E$ con respecto a $\lambda$
%    \begin{equation}
%        \frac{dE}{d\lambda} = 0 \Rightarrow 
%        \lambda^2_{min}=\frac{2^{1/2}}{\hbar}k^{1/2}m^{1/2}\bigg(\frac{\pi^2}{24}-\frac{1}{4}\bigg)^{1/2}
%    \end{equation}
%    
%    \item Calcular la energía para la mejor aproximación al
%    estado fundamental.
%    \begin{equation}
%        E(\lambda_{min})=1.14\frac{1}{2}\hbar\sqrt{\frac{k}{m}}
%    \end{equation}
%    En este ejemplo, dado que la energía del oscilador armónico
%    es $1/2\hbar\omega$ estamos sobreestimando la energía en un
%    14 \%.
%\end{enumerate}

\subsection{El método de variaciones lineal}
Supongamos ahora que nuestra función de prueba es una
suma de funciones reales linealmente independientes,
$\{\varphi_1$, $\varphi_2$, ..., $\varphi_N\}$, que
pueden ser o no ortonormales, cada una de ellas con
un coeficiente $c_j$,
\begin{equation}
    \phi=\sum_{j=1}^{N}c_j\varphi_j
\end{equation}
y que lo que queremos optimizar son los valores de los 
coeficientes $c_j$. Tenemos por tanto más de un parámetro 
variacional. El método que acabamos de ver puede volverse 
algo tedioso para resolver este problema. Afortunadamente,
hay una alternativa sistemática, el método de variaciones lineal. 
 
Dado que tenemos una suma de términos en nuestra función de
prueba, podemos calcular la energía a partir de la Ecuación
\ref{eq:evar} de tal manera que la energía dependerá de los
coeficientes $c_k$
\begin{equation}
    E=\frac{\sum_{j=1}^N\sum_{i=1}^Nc_jc_iH_{ji}}{\sum_{j=1}^N\sum_{i=1}^N c_j c_iS_{ji}}
\end{equation}
donde estamos usando la \textbf{integral de solapamiento}
$S_{ji}=\int \varphi^\star\varphi d\tau=S^\star_{ij}=S_{ij}$.
Reorganizando los términos de esta expresión podemos obtener 
\begin{equation}
    E\sum_{j=1}^N\sum_{i=1}^N c_j c_iS_{ji}=\sum_{j=1}^N\sum_{i=1}^Nc_jc_iH_{ji}
\end{equation}
De acuerdo con el principio variacional, la condición que nos 
permite llegar a la combinación lineal de mínima energía es
\begin{equation}
    \frac{\partial E}{\partial c_k}=0
    \label{eq:var_deriv}
\end{equation}

Dado que el método variacional lineal se vuelve complicado
incluso para sistemas muy pequeños, trabajaremos con el ejemplo
más sencillo, correspondiente a un sistema que aproximamos
con una función de onda 
\begin{equation}
    \psi=c_1\phi_1 + c_2\phi_2
\end{equation}
Si desarrollamos la integral variacional, obtenemos lo siguiente
\begin{equation}
\begin{split}
    E&=\frac{\int\psi^\star\hat{H}\psi d\tau}{\int\psi^\star\psi d\tau}  
    = \frac{\int( c_1\phi_1+ c_2\phi_2)\hat{H}(c_1\phi_1+ c_2\phi_2)d\tau}{\int (c_1\phi_1+ c_2\phi_2)(c_1\phi_ 1+ c_2\phi_2)d\tau}= \\
    &= \frac{c_1^2\int\phi_1\hat{H}\phi_1 d\tau +
    c_2^2\int\phi_2\hat{H}\phi_2 d\tau  +
    2c_1c_2\int\phi_1\hat{H}\phi_2 d\tau}
    {c_1^2\int\phi_1\phi_1 d\tau  +
    c_2^2\int\phi_2\phi_2 d\tau  +
    2c_1c_2\int\phi_1\phi_2 d\tau} = \\
    & = \frac{c_1^2H_{11} + c_2^2H_{22} + 2c_1c_2H_{12}}
    {c_1^2 + c_2^2 + 2c_1c_2S_{12}}
    \end{split}
\end{equation}
Podemos reorganizar esta ecuación para la energía como
\begin{equation}
    E(c_1^2S_{11} + c_2^2S_{22} + 2c_1c_2S_{12})=c_1^2H_{11} + c_2^2H_{22} + 2c_1c_2H_{12}
    \label{eq:var_exp}
\end{equation}
En este caso, al tener únicamente dos funciones de onda, 
la Ecuación \ref{eq:var_deriv} pasa a ser
\begin{align}
    \frac{\partial E}{\partial c_1}&=0       & \frac{\partial E}{\partial c_2}&=0
\end{align}
Podemos por tanto derivar la Ecuación \ref{eq:var_exp}
con respecto a $c_1$ y $c_2$, y así obtenener
\begin{align}
 E(2c_1S_{11} +2c_2S_{12})    &=2c_1H_{11} +2c_2H_{12} \label{eq:seq1}\\
 E(2c_2S_{22} +2c_1S_{12})    &=2c_2H_{22} +2c_1H_{12}
 \label{eq:seq2}
\end{align}
Estas dos ecuaciones forman un sistema de ecuaciones 
lineales cuyas variables son $c_1$ y $c_2$, que 
podemos reescribir en forma matricial
\begin{equation}
\begin{pmatrix}
 H_{11}-ES_{11} &  H_{12}-ES_{12}  \\ 
 H_{21}-ES_{21} &  H_{22}-ES_{22}
\end{pmatrix}
\begin{pmatrix}
c_{1}  \\ 
c_{2}
\end{pmatrix}=
\begin{pmatrix}
0\\
0
\end{pmatrix}
\end{equation}
Este sistema tiene una solución no trivial, es decir,
una solución que no sea $c_1=c_2=0$, sólo cuando resolvemos 
el \textbf{determinante secular}
\begin{equation}
\begin{vmatrix}
 H_{11}-ES_{11} &  H_{12}-ES_{12}  \\ 
 H_{21}-ES_{21} &  H_{22}-ES_{22}
\end{vmatrix}= 0
\end{equation}
El determinante secular arroja dos valores posibles
para la energía. Entre ellas, la más pequeña es 
la que nos quedaremos como aproximación para la energía
del estado fundamental. A partir de la energía, y usando
las Ecuaciones \ref{eq:seq1} y \ref{eq:seq2} podemos
encontrar los coeficientes $c_1$ y $c_2$ que permiten
recuperar la función de onda variacional.

\section{Los métodos perturbativos}
Otra posibilidad es que nos enfrentemos a un problema 
para el que no sepamos resolver la ecuación de Schrödinger,
$\hat{H}\psi=E\psi$, pero que sí sepamos resolverla para
un problema similar, pero más sencillo.
Si es así, podemos aproximar el hamiltoniano como suma 
de un término fundamental, $H^{(0)}$, cuyas funciones propias 
$\psi^{(0)}$ y valores propios $E^{(0)}$ son conocidos,
más otros términos, que llamamos \textbf{perturbación}, 
\begin{equation}
    \hat{H}=\hat{H}^{(0)} + \lambda\hat{H}^{(1)} + 
    \lambda^2 \hat{H}^{(2)} + ...
    \label{eq:H_pert}
\end{equation}
En esta ecuación, los términos $\hat{H}^{(1)}$ y 
$\hat{H}^{(2)}$ representan la diferencia entre el
hamiltoniano real y el modelo. En general,
$\lambda \ll 1$, de modo que las potencias
$\lambda^i\hat{H}^{(i)}$ decrecen sustancialmente 
a medida que aumenta $i$. En los desarrollos que 
incluimos a continuación, consideramos únicamente los
términos correspondientes a $\hat{H}^{(0)}$ y 
$\hat{H}^{(1)}$.

Del mismo modo que hemos hecho para el hamiltoniano, 
podemos escribir las funciones de onda para el estado $k$
del sistema que nos interesa describir y su energía 
como suma de un término fundamental y sus perturbaciones
\begin{align}
    \psi =& \psi_k^{(0)} + \lambda\psi_k^{(1)} + \lambda^2\psi_k^{(2)} + ... \label{eq:psi_pert}\\
    E =& E_k^{(0)} + \lambda E_k^{(1)} + \lambda^2E_k^{(2)}+ ...
    \label{eq:E_pert}
\end{align}
En esta ecuación, $E_k^{(1)}$ es la corrección de primer orden
para la energía,  $E_k^{(2)}$ es la corrección de segundo orden 
para la energía del
estado $k$, etc. 

Como sabemos que podemos obtener la energía
a partir de $\hat{H}\psi = E\psi$, podemos desarrollar la 
expresión del hamiltoniano en función de los términos de las Ecuaciones
\ref{eq:H_pert}, \ref{eq:psi_pert} y \ref{eq:E_pert}:
\begin{equation}
\begin{split}
    \hat{H}\psi =& (\hat{H}^{(0)} + \lambda \hat{H}^{(1)} + ...)(\psi^{(0)} + \lambda\psi^{(1)} +  + \lambda\psi^{(2)} + ...) = \\
    =& \hat{H}^{(0)}\psi^{(0)} + 
    \lambda (\hat{H}^{(1)}\psi^{(0)} + \hat{H}^{(0)}\psi^{(1)}) + \\ 
    &+ \lambda^2 (\hat{H}^{(2)}\psi^{(0)} + \hat{H}^{(1)}\psi^{(1)} + \hat{H}^{(2)}\psi^{(0)} ) =  \\
    =& E^{(0)}\psi^{(0)} + \lambda (E^{(1)}\psi^{(0)} + E^{(0)}\psi^{(1)}) + \\
    &+ \lambda^2 (E^{(2)}\psi^{(0)} + E^{(1)}\psi^{(1)} + E^{(2)}\psi^{(0)})
\end{split}
\end{equation}
Podemos agrupar los términos en esta ecuación en función del
exponente de $\lambda$ para obtener
\begin{align}
    \lambda^0\mathrm{:} & \hat{H}^{(0)}\psi^{(0)}= E^{(0)}\psi^{(0)}\label{eq:HE_pert0}\\
    \lambda^1\mathrm{:}&
    \hat{H}^{(1)}\psi^{(0)} + \hat{H}^{(0)}\psi^{(1)} = E^{(1)}\psi^{(0)} + E^{(0)}\psi^{(1)} \label{eq:HE_pert1}\\
    \lambda^2\mathrm{:}  &     
     \hat{H}^{(0)}\psi^{(2)} + \hat{H}^{(1)}\psi^{(1)} = E^{(2)}\psi^{(0)} + E^{(1)}\psi^{(1)}+ E^{(0)}\psi^{(2)}
\end{align}

A partir de este conjunto de ecuaciones, podemos empezar a trabajar
con la función de onda del estado fundamental, $\psi_0$. La Ecuación
\ref{eq:HE_pert0} pasa a ser
\begin{equation}
    \hat{H}^{(0)}\psi_0^{(0)} = E_0^{(0)}\psi_0^{(0)}
\end{equation}
donde tenemos la energía y la función de onda conocidas. La segunda ecuación
, correspondiente al exponente $\lambda^1$ (Ec. \ref{eq:HE_pert1}), 
es 
\begin{equation}
    \hat{H}^{(1)}\psi_0^{(0)} + \hat{H}^{(1)}\psi_0^{(1)} = 
    E_0^{(0)}\psi_0^{(1)} + E_0^{(1)}\psi_0^{(0)}
\end{equation}
Podemos expresar la perturbación de la función de onda $\psi_0^{(1)}$ 
como combinación lineal de las funciones propias de $\hat{H}^{(0)}$,
de modo que $\psi_0^{(1)}=\sum_nc_n\psi_n^{(0)}$. Sustituyendo
dentro de la expresión anterior, obtenemos
\begin{equation}
    \hat{H}^{(1)}\psi_0^{(0)} + \sum_nc_n\hat{H}^{(0)}\psi_n^{(0)} = 
    \sum_nc_nE_0^{(0)}\psi_n^{(0)} + E_0^{(1)}\psi_0^{(0)}
\end{equation}
Si multiplicamos esta ecuación por la izquierda por $\psi_0^{(0)}^\star$
e integramos llegamos a
\begin{equation}
\begin{split}
  \int \psi_0^{(0)}^\star \hat{H}^{(1)}\psi_0^{(0)}d\tau +  \sum_nc_n\int \psi_0^{(0)}^\star\hat{H}^{(0)}\psi_n^{(0)}d\tau = \\
    = \sum_nc_nE_0^{(0)}\int \psi_0^{(0)}^\star\psi_n^{(0)}d\tau + \int \psi_0^{(0)}^\star E_0^{(1)}\psi_0^{(0)}d\tau
\end{split}
\end{equation}
Dada la condición de ortonormalidad, el segundo término a la izquierda de
la igualdad y el primero por la derecha serán $0$ para $n\neq 0$. Por otra
parte, dada la condición de normalización, el último término es igual a 
$E_0^{(1)}$. De este modo llegamos a una expresión compacta para el
valor promedio de la perturbación que obtenemos a partir de la función 
de onda sin perturbar
\begin{equation}
    E_0^{(1)}= \int \psi_0^{(0)}^\star \hat{H}^{(1)}\psi_0^{(0)}d\tau 
\end{equation}

Haciendo otra manipulación, podemos obtener los valores de los 
coeficientes $c_k$ de la combinación lineal. Si multiplicamos la
Ecuación \ref{eq:HE_pert1} por $\psi_k^{(0)}^\star$ para $k\neq 0$,
obtenemos
\begin{equation}
    \begin{split}
        \int \psi_k^{(0)}^\star \hat{H}^{(1)}\psi_0^{(0)}d\tau +  \sum_nc_n\int \psi_k^{(0)}^\star\hat{H}^{(0)}\psi_n^{(0)}d\tau = \\
    = \sum_nc_nE_0^{(0)}\int \psi_k^{(0)}^\star\psi_n^{(0)}d\tau + \int \psi_k^{(0)}^\star E_0^{(1)}\psi_0^{(0)}d\tau
    \end{split}
\end{equation}
Esta expresión se puede simplificar usando las condiciones de
ortogonalidad y normalización de las funciones de onda 
$\{\psi_n^{(0)}\}$ para alcanzar la siguiente ecuación:
\begin{equation}
    c_k=-\frac{\int \psi_k^{(0)}\hat{H}^{(1)}\psi_0^{(0)}}{E_k^{(0)} - E_0^{(0)}}
\end{equation}

Finalmente, aunque no haremos esta derivación de manera explícita,
a partir de los términos correspondientes a $\lambda^2$
podemos derivar la expresión para la corrección de segundo orden de
la energía
\begin{equation}
    E_0^{(2)} = \sum_{n\neq 0}\frac{|\int\psi_n^{(0)}^\star\hat{H}^{(1)}\psi_0^{(0)}d\tau|^2}{E_0^{(0)} - E_n^{(0)}} = 
    \sum_{n\neq 0}\frac{|H_{n0}^{(1)}|^2}{E_0^{(0)} - E_n^{(0)}}
\end{equation}

%\section{Método de Rayleigh-Schrödinger}
%Este método permite obtener $\psi^{(i)}_k$ como combinación 
%lineal de las funciones funciones propias sin 
%perturbar del hamiltoniano $\hat{H}^{(0)}$ (funciones de orden cero ):
%\begin{equation}
%    \psi_k^{(i)} = \sum_{j=0}^\infty \psi_j^{(0)}c_{kj}^{(i)}
%\end{equation}
%El subíndice $k$ en $c_{kj}^{(i)}$ representa el estado estacionario
%que tratamos de corregir, $(i)$ es el orden de la perturbación y $j$
%las funciones de orden cero para todos los estados estacionarios.
%Sustituyendo la expresión para la función de onda en la Ecuación
%\ref{eq:H_pert} obtenemos
%\begin{equation}
%    \sum c_{kj}(\hat{H}^{(0)}-E_k^{(0)})=
%    -(\hat{H}^{(1)}-E_k^{(1)}))\psi_k^{(0)}
%\end{equation}
%Hay dos casos a considerar.
%\begin{enumerate}
%    \item Caso en el que $n=k$:
%    \begin{equation}
%        E_n^{(1)} = \int \psi_n^{(0)}^\star \hat{H}^{(1)}\psi_n^{(0)} d\tau
%    \end{equation}
%    En este caso la corrección de primer orden se obtiene promediando
%    la perturbación $\hat{H}^{(1)}$ sobre las correspondientes
%    funciones de onda sin perturbar. Para evaluar la corrección de 
%    primer orden a la energía basta con conocer la función de onda
%    sin perturbar. Por tanto, la calcularemos usando la 
%    expresión
%    \begin{equation}
%        E_n\simeq E_n^{(0)} +  E_n^{(1)} = E_n^{(0)} +  \int \psi_n^{(0)}^\star \hat{H}^{(1)}\psi_n^{(0)} d\tau
%    \end{equation}
%
%    \item Caso en el que $n\neq k$: 
%    \begin{equation}
%        c_{kn} = -\frac{\int \psi_n^{(0)}^\star {\hat{H}}^{(1)}\psi_k^{(0)} d\tau}
%        {E_k^{(0)} - E_n^{(0)}}
%    \end{equation}
%    En este caso los coeficicentes $c_{kn}$ se calculan usando esta
%    ecuación. Para corregir a primer orden la energía de un estado
%    sólo necesitamos evaluar $\hat{H}^{(1)}$ en este estado y
%    conocer la función de onda sin perturbar. Las correcciones
%    se vuelven más y más complicadas al aumentar el orden de la
%    perturbación.
%\end{enumerate}
%Las ecuaciones que hemos introducido proporcionan la 
%siguiente expresión para la corrección de primer 
%orden de la función de onda:
%\begin{equation}
%    \psi_{k}^{(1)} = \sum_{n\neq k}\frac{\int {\psi_n^{(0)}}^\star \hat{H}^{(1)}\psi_k^{(0)} d\tau}{E_k^{(0)} - E_n^{(0)}}
%\end{equation}
%Haciendo $\lambda=1$ usando solamente la corrección de primer orden de la
%función de onda obtenemos como aproximación a la función de onda 
%perturbada
%\begin{equation}
%    \psi_k = \psi_k^{(0)} +  
%    \sum_{n\neq k}\frac{\int {\psi_n^{(0)}}^\star\hat{H}^{(1)}\psi_k^{(0)} d\tau}
%    {E_k^{(0)} - E_n^{(0)}}\psi_n^{(0)}
%\end{equation}

\subsection{Soluciones variacional y perturbativa del átomo del Helio}
En el próximo tema estudiaremos de manera sistemática 
los átomos polielectrónicos. Sin embargo, para ver cómo 
podemos utilizar los métodos aproximados que hemos introducido
aquí para resolver
problemas realistas, veremos su aplicación al
átomo de Helio, el más sencillo de los sistemas que
implican más de dos partículas. 

Bajo la aproximación de masa
nuclear infinita ($m_N/m_e\rightarrow \infty$)
el hamiltoniano para el átomo de Helio es
\begin{equation}
    \hat{H} = -\frac{\hbar^2}{2m_{e1}}\nabla^2_1 
    - \frac{2e^2}{4\pi\varepsilon_0r_1} 
    -\frac{\hbar^2}{2m_{e2}}\nabla^2_2 
    - \frac{2e^2}{4\pi\varepsilon_0r_2} 
            + \frac{e^2}{4\pi\varepsilon_0r_{12}}
\end{equation}
En el lado derecho de esta ecuación encontramos, en
primer lugar, los términos de energía
cinética y potencial correspondientes a los dos electrones
con respecto al núcleo. El último término corresponde a
la repulsión interelectrónica, donde $r_1$ y $r_2$ son las
distancias entre el núcleo y los electrones, y 
$r_{12}=|\mathbf{r}_2 - \mathbf{r}_1|$ 
es la distancia interelectrónica.

La manera más drástica de solventar este problema 
sería ignorar completamente la repulsión entre los electrones
1 y 2. Si truncásemos el hamiltoniano, excluyendo su último
término, el problema sería perfectamente resoluble, dado
que sería equivalente a tener dos electrones para un átomo
hidrogenoide con $Z=2$,
\begin{equation}
\hat{H} = \hat{H}_1+ \hat{H}_2
\end{equation}
Si escribimos nuestra función de onda como
el producto de dos funciones 1s,
\begin{equation}
    \psi(\mathbf{r}_1, \mathbf{r}_2)=\psi_{1s}(\mathbf{r}_1)\psi_{1s}(\mathbf{r}_2)
\end{equation}
entonces la energía será, simplemente
\begin{equation}
    E = E_1 + E_2 
\end{equation}
En el caso del Helio, nuestra estimación de la energía
como suma de dos átomos hidrogenoides, sería -4$E_h$, 
donde $E_h$ es la energía del estado fundamental
del átomo de hidrógeno (i.e.
$E_h=m_ee^4/(16\pi^2\varepsilon_0^2\hbar^2$). El valor 
experimental es solo -2.903$E_h$, con lo cual esta aproximación
claramente sobrestima la estabilidad del Helio, como
esperábamos al ignorar la interacción repulsiva
de los dos electrones.

Usando el método variacional simple, podemos intentar mejorar 
esta aproximación. Para ello, podemos usar como funciones
de prueba expresiones del tipo
\begin{equation}
    \phi(\mathbf{r}_1, \mathbf{r}_2)=\psi_{1s}(\mathbf{r}_1)\psi_{1s}(\mathbf{r}_2)
    \label{eq:varphi}
\end{equation}
donde ahora los orbitales atómicos dependen de un 
parámetro variacional, $Z$
\begin{equation}
    \psi_\mathrm{1s}(r_j)=\bigg(\frac{\zeta^3}{\pi a_0}\bigg)^{1/2}e^{-\zeta r_j/a_0}
\end{equation}
Usando la expresión de la Ecuación \ref{eq:varphi} y el 
hamiltoniano podemos obtener la energía en función de $\zeta$,
$E(\zeta)=\frac{m_ee^4}{16\pi^2\epsilon_0^2\hbar^2}$.
De acuerdo con el método variacional simple, la mejor aproximación
corresponderá al valor de $\zeta$ que minimice la energía, donde
$dE/d\zeta=0$. Realizando este cálculo, el valor de la energía
resultante es -2.848$E_h$, mucho más próximo al valor 
experimental de -2.903$E_h$. Vemos así que el uso de una
función de prueba es enormemente provechoso para aproximar
la energía. Además, la solución resultante tiene significado
físico: el valor óptimo para $\zeta$ es 1.69, un valor algo
inferior a 2, que refleja el apantallamiento de la carga del
núcleo del átomo de Helio por parte de los electrones. 

Hay muchas otras funciones que pueden usarse como función de
prueba para intentar mejorar esta solución, entre las cuales
son particularmente populares los \textbf{orbitales de tipo Slater},
descritos como
\begin{equation}
    S_{nlm}(r, \theta, \phi)=N_{nl}r^{n-1}e^{-\zeta r}Y_l^{ml}(\theta, \phi)
\end{equation}
Aquí, $N_{nl}$ es la constante de normalización y $Y_l^{ml}(\theta, \phi)$ son los armónicos esféricos.

Finalmente, también podemos usar el método perturbativo para
abordar el problema que presenta el átomo de Helio. En este caso, 
definimos $\hat{H}^{(0)}$ como el hamiltoniano para un solo electrón, 
mientras que la perturbación $\hat{H}^{(1)}$ corresponderá a la
repulsión interelectrónica, 
\begin{equation}
\hat{H}^{(1)}=\frac{e^2}{4\pi \varepsilon_0r_{12}}
\end{equation}
En este caso, podemos evaluar la perturbación de primer orden usando
la integral
\begin{equation}
E^{(1)}=\frac{e^2}{4\pi \epsilon_0}\int\int d\mathbf{r}_1d\mathbf{r}_2\psi_{1s}^\star(\mathbf{r}_1)\psi_{1s}^\star(\mathbf{r}_2)\frac{1}{r_{12}}\psi_{1s}(\mathbf{r}_1)\psi_{1s}(\mathbf{r}_2)
\end{equation}
Para poder realizar esta integral con 
respecto a $\mathbf{r}_1$ y $\mathbf{r}_2$, debemos sustituir 
$r_{12}=(r_1+r_2-2r_1r_2\cos\theta)^{1/2}$
La estimación usando la perturbación de primer orden para la energía
calculada con este método es -2.75$E_h$, e incluyendo los 
términos de segundo orden, es -2.908$E_h$. Así, vemos que también 
el método perturbativo pueden proporcionar 
muy buenas estimaciones de la energía exacta del sistema a pesar
de carecer de soluciones analíticas.

\begin{table}[]
    \centering
    \begin{tabular}{ | l | c |}
    \hline
        \textbf{Método} & \textbf{Energía/$E_h$} \\
         \hline
         Experimento &  -2.9033 \\
         Ignorando $H_{12}$ & -4.0000 \\
         Variacional & -2.8477 \\
         Perturbación 1$^\mathrm{er}$ orden & -2.7500 \\
         Perturbación 2º orden & -2.9077 \\
     \hline    
    \end{tabular}
    \caption{Estimaciones de la energía del átomo de 
    Helio a partir de  diferentes aproximaciones.}
    \label{tab:my_label}
\end{table}

%\begin{thebibliography}{}
%\bibitem{atkins_depaula} Atkins, P and De Paula, J. 2006. ``Physical Chemistry, 8th Edition''. Oxford University Press.
%
%\bibitem{atkins} Atkins, P and Friedman, R. 2005. ``Molecular Quantum Mechanics, 4th Edition''. Oxford University Press.
%
%\bibitem{fleisch} Fleisch, Daniel A. 2020. ``A student's guide to the Schr{\"o}dinger equation''. Cambridge University Press
%
%\bibitem{levine} Levine, I. N. 2013. ``Quantum
%Chemistry, 7th Edition''. Pearson.
%\end{thebibliography}
\bibliographystyle{plainnat}



\end{document}
