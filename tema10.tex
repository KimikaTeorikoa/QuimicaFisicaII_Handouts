\documentclass{tufte-handout}

%\geometry{showframe}% for debugging purposes -- displays the margins

\usepackage[spanish, es-tabla]{babel}
\usepackage{amsmath}
\newtheorem{theorem}{Postulado}
\usepackage{mathabx}
\usepackage{physics}
\usepackage{amsmath}
% Set up the images/graphics package
\usepackage{graphicx}
\setkeys{Gin}{width=\linewidth,totalheight=\textheight,keepaspectratio}
\graphicspath{{graphics/}}

\title[Química Física II: Tema 10 - Moléculas poliatómicas]{
Tema 10: Estructura Electrónica de Moléculas Poliatómicas}
%\author{David De Sancho}
\date{}  % if the \date{} command is left out, the current date will be used

% The following package makes prettier tables.  We're all about the bling!
\usepackage{booktabs}

% The units package provides nice, non-stacked fractions and better spacing
% for units.
\usepackage{units}

% The fancyvrb package lets us customize the formatting of verbatim
% environments.  We use a slightly smaller font.
\usepackage{fancyvrb}
\fvset{fontsize=\normalsize}

% Small sections of multiple columns
\usepackage{multicol}

% Provides paragraphs of dummy text
\usepackage{lipsum}

% These commands are used to pretty-print LaTeX commands
\newcommand{\doccmd}[1]{\texttt{\textbackslash#1}}% command name -- adds backslash automatically
\newcommand{\docopt}[1]{\ensuremath{\langle}\textrm{\textit{#1}}\ensuremath{\rangle}}% optional command argument
\newcommand{\docarg}[1]{\textrm{\textit{#1}}}% (required) command argument
\newenvironment{docspec}{\begin{quote}\noindent}{\end{quote}}% command specification environment
\newcommand{\docenv}[1]{\textsf{#1}}% environment name
\newcommand{\docpkg}[1]{\texttt{#1}}% package name
\newcommand{\doccls}[1]{\texttt{#1}}% document class name
\newcommand{\docclsopt}[1]{\texttt{#1}}% document class option name

\begin{document}

\maketitle% this prints the handout title, author, and date

\begin{abstract}
\noindent 
En este tema extendemos los conceptos que aplicamos en
el caso de las moléculas sencillas a moléculas que tienen
más de dos átomos. En particular, nos centraremos en la 
teoría de enlace de valencia para las moléculas de
agua y metano, que nos permitirán entender 
conceptos clave como la promoción de electrones y la
hibridación.
\end{abstract}

%\printclassoptions



\section{}

Para moléculas poliatómicas, construimos orbitales
moleculares tal y como los definimos en el caso de
las diatómicas, pero usando un mayor número de
orbitales atómicos. Así, un orbital molecular $\psi$
tiene la forma general
\begin{equation}
    \psi = \sum_ic_i\psi_i^{OA}
\end{equation}
donde $\psi_i^{OA}$ es un orbital atómico y la
suma se extiende sobre todos los orbitales de valencia
de todos los átomos que forman la molécula.
Debido al mayor número de orbitales, aumenta el
número de geometrías posibles. A pesar de esta mayor
complejidad, la teoría nos permite predecir la
forma de una molécula poliatómica y especificar las
longitudes y ángulos de enlace, mediante la 
determinación de cuál de entre las diferentes 
configuraciones es la de menor energía. 

\section{Deficiencias de la teoría}
En el caso de la molécula de agua, los orbitales
disponibles son 2s$^2$2p$_x^2$2p$_y^1$2p$_z^1$
por parte del átomo de oxígeno y 1s$^1$ por parte
del hidrógeno. Para formar H$_2$O, cada electrón
no apareado de los orbitales 2p del oxígeno se
puede apararear con un orbital H1s. En cada
combinación se forma un enlace $\sigma$, de
simetría cilíndrica alrededor del eje internuclear,
O-H. Así, podríamos predecir que la que la
molécula de agua es angular, y que el ángulo 
formado por los tres átomos sería de 90º. Sin embargo,
sabemos que el ángulo enlazante real no es de 90º sino
de 104.5º.

Otra deficiencia de la teoría de enlace de valencia
es su incapacidad para explicar cómo el carbono
forma cuatro enlaces, ya que su configuración es 
1s$^2$2s$^2$2p$^1_x$2p$_y^1$. Este átomo debería 
ser capaz de formar sólo dos enlaces, en lugar de
cuatro.

\section{Promoción e hibridación}
La promoción es la excitación de un electrón 
a un orbital de mayor energía. Por ejemplo, en
el caso del carbono, un electrón podría promocionar
desde el orbital 2s al 2p$_z$, para así pasar a tener
cuatro electrones no apareados en la configuración
2s$^1$2p$_x^1$2p$_y^1$2p$_z^1$. Esto explicaría la
formación de moléculas como el metano, al combinarse
estos orbitales de valencia con cuatro orbitales 1s 
de cuatro átomos de hidrógeno. La promoción requiere
energía, pero este requerimiento lo compensa la
capacidad del átomo promovido para formar 
cuatro enlaces en lugar de dos.

Sin embargo, esta descripción resultaría en cuatro 
enlaces diferentes entre sí, lo cual se contradice
con la simetría de los cuatro enlaces C-H. El 
concepto de hibridación resuelve este nuevo problema.
La hibridación es una combinación lineal de orbitales
atómicos dentro del mismo átomo para formar otros que 
nos permitan explicar de forma simple todos los tipos 
de enlace que ocurren en las moléculas poliatómicas.

\subsection{Metano}
En el caso del metano, tendríamos cuatro orbitales 
híbridos
\begin{align*}
    h_1=s+p_x+p_y+p_z\\
    h_2=s-p_x+p_y-p_z\\
    h_3=s-p_x-p_y+p_z\\
    h_4=s+p_x-p_y-p_z
\end{align*}
Se trata de una hibridación sp$^3$ en la que los ejes 
de los orbitales tienen simetría tetraédrica, con
un ángulo de 109.47º. Estos orbitales son capaces de
formar cuatro enlaces $\sigma_\mathrm{C-H}$ equivalentes
formados por el orbital híbrido $h_i$ y un orbital 
H1s. La función de onda del orbital molecular correspondiente
será $\sigma_\mathrm{C-H}= h_1(1)1s_{A}(2)+h_1(2)1s_{A}(1)$.

\subsection{Eteno}
En el caso de la molécula de eteno, la hibridación explica
la formación de un doble enlace H$_2$C=CH$_2$. Se trata de 
una molécula plana con ángulos entre los átomos próximos
a los 120º. En este caso un electrón del orbital 2s promueve
a uno  de los orbitales p y se forman tres orbitales híbridos
sp$^2$, ubicados en un plano y orientados hacia los vértices
de un triángulo
\begin{align*}
    h_1&=s+2^{1/2}p_y\\
    h_2&=s+(3/2)^{1/2}p_x-(1/2)^{1/2}p_y\\
    h_3&=s-(3/2)^{1/2}p_x-(1/2)^{1/2}p_y
\end{align*}
El tercer orbital, 2p$_z$ no está incluido
en la hibridación y su eje es perpendicular al plano del
enlace.  La formación de enlaces $\sigma$ se produce al aparearse
los orbitales sp$^2$ de dos átomos de carbono y de estos
con los orbitales 1s de cuatro átomos de hidrógeno. Además,
se forman enlaces $\pi$ a partir de los orbitales atómicos
p$_z$ de los dos átomos de carbono.

\subsection{Etino}
Finalmente, en el caso de la molécula de etino, se forman
orbitales híbridos sp, a partir del s y el p$_z$
\begin{align*}
    h_1&=s+p_z\\
    h_2&=s-p_z
\end{align*}
Los orbitales híbridos resultantes están dirigidos en sentidos
opuestos del eje entre los dos núcleos de la molécula, que
permiten la formación de enlaces tipo $\sigma$ tanto entre
los dos carbonos como con sendos átomos de hidrógeno
a partir de sus orbitales 1s. Los electrones que ocupan orbitales
p libres perpendiculares al eje de la molécula se aparean
para formar enlaces $\pi$.

\subsection{Benceno}
Concluimos nuestra exposición de las moléculas poliatómicas
con el benceno, cuyos seis átomos de carbono con hibridación 
sp$^2$ se unen entre sí mediante enlaces $\sigma$. Asimismo,
cada uno de los átomos de carbono se enlaza además a un átomo 
de hidrógeno mediante un enlace $\sigma$ combinándose con los
correspondientes orbitales 1s. La densidad electrónica es 
idéntica en los seis enlaces C-C, lo cual resulta en una geometría
plana en forma de hexágono. Todos los enlaces C-C tienen la 
misma longitud (1.39 {\AA} y todos los ángulos de enlace 
son de 120º. Como los átomos de carbono presentan hibridación
sp$^2$, cada átomo de carbono tiene un orbital p perpendicular
al plano del anillo que solapa con los orbitales p de los 
carbonos contiguos para formar un círculo de densidad electrónica 
p por encima y por debajo del plano molecular.
La representación del benceno como un hexágono regular con un 
círculo en el centro evoca el solapamiento cíclico de los seis
orbitales 2p.

Se puede representar por dos estructuras de Lewis, que difieren 
entre sí en la disposición de los electrones que forman enlaces 
$\pi$. La molécula real es un híbrido de todas ellas y, cada
estructura se llama forma resonante. La molécula de benceno es 
el prototipo de molécula aromática con una estabilidad extra 
debida a la deslocalización.


%\begin{thebibliography}{}
%\bibitem{atkins_depaula} Atkins, P and De Paula, J. 2006. ``Physical Chemistry, 8th Edition''. Oxford University Press.
%
%\bibitem{atkins} Atkins, P and Friedman, R. 2005. ``Molecular Quantum Mechanics, 4th Edition''. Oxford University Press.
%
%\bibitem{fleisch} Fleisch, Daniel A. 2020. ``A student's guide to the Schr{\"o}dinger equation''. Cambridge University Press
%
%\bibitem{levine} Levine, I. N. 2013. ``Quantum
%Chemistry, 7th Edition''. Pearson.
%\end{thebibliography}
\bibliographystyle{plainnat}



\end{document}
