\documentclass{tufte-handout}

%\geometry{showframe}% for debugging purposes -- displays the margins

\usepackage[spanish, es-tabla]{babel}
\usepackage{amsmath}

% Set up the images/graphics package
\usepackage{graphicx}
\setkeys{Gin}{width=\linewidth,totalheight=\textheight,keepaspectratio}
\graphicspath{{graphics/}}

\title[Química Física II: Ejercicios 1$^{er}$ Set]{
Química Física II: Ejercicios 1$^{er}$ Parcial}
%\author{David De Sancho}
\date{}  % if the \date{} command is left out, the current date will be used

% The following package makes prettier tables.  We're all about the bling!
\usepackage{booktabs}

% The units package provides nice, non-stacked fractions and better spacing
% for units.
\usepackage{units}

% The fancyvrb package lets us customize the formatting of verbatim
% environments.  We use a slightly smaller font.
\usepackage{fancyvrb}
\fvset{fontsize=\normalsize}

% Small sections of multiple columns
\usepackage{multicol}

% Provides paragraphs of dummy text
\usepackage{lipsum}

% These commands are used to pretty-print LaTeX commands
\newcommand{\doccmd}[1]{\texttt{\textbackslash#1}}% command name -- adds backslash automatically
\newcommand{\docopt}[1]{\ensuremath{\langle}\textrm{\textit{#1}}\ensuremath{\rangle}}% optional command argument
\newcommand{\docarg}[1]{\textrm{\textit{#1}}}% (required) command argument
\newenvironment{docspec}{\begin{quote}\noindent}{\end{quote}}% command specification environment
\newcommand{\docenv}[1]{\textsf{#1}}% environment name
\newcommand{\docpkg}[1]{\texttt{#1}}% package name
\newcommand{\doccls}[1]{\texttt{#1}}% document class name
\newcommand{\docclsopt}[1]{\texttt{#1}}% document class option name

\begin{document}
\maketitle
\large
\begin{enumerate}
    \item La función trabajo (\textit{work
    function}) para el potasio es de 2.24
    eV. Si se ilumina el potasio metálico
    con luz de longitud de onda de 480 nm,
    calcula (a) la energía cinética máxima
    de los fotoelectrones y (b) la longitud
    de onda umbral.
    
    \item Se lanza una roca con masa de 
    50 g con velocidad de 40 m/s. ¿Cuál es
    su longitud de onda de De Broglie?
    
    \item Se mide la velocidad de un
    electrón y resulta ser de $5\times10^3$
    m/s $\pm$0.003\%. ¿Dentro de qué
    límites puede conocerse la posición de
    dicho electrón a lo largo de la
    dirección del vector velocidad?
    
    \item Suponiendo que $\psi_1$ y
    $\psi_2$ son dos funciones reales que 
    están normalizadas y son ortogonales, 
    normaliza las siguientes funciones:
    (a) $\psi_1+\psi_2$, (b) 
    $\psi_1-\psi_2$.
    
    \item ¿Cuál de las siguientes funciones
    es función propia del operador
    $\hat{D}=\frac{d}{dx}$? (a) $kx^2$
    (b) $\sin x$ (c) $\exp (kx)$ 
    (d) $\exp(kx^2)$ (e) $\exp(ikx)$.
    
    \item Los operadores para la posición y
    el momento lineal vienen dados por 
    \begin{equation}
        \begin{array}{cc}
             \hat{x}& =x  \\
            \hat{p}_x& =\frac{\hbar}{i}\big(\frac{\partial}{\partial x}\big)
        \end{array}
    \end{equation}
    (a) ¿Son estos operadores hermíticos?
    (b) ¿Conmutan estos operadores?
    (c) ¿Son los operadores citados lineales
    o no lineales?
    (d) ¿Es la función $\psi_1=A\sin (n\pi
    x/a)$, donde $A$, $n$ y $a$ son
    constantes, una función propia de ambos
    operadores?
   
   \item Normaliza la función del ejercicio
   6d en el dominio $0\leq x\leq a$.
   
   \item Determina las condiciones necesarias 
   para que la función $\psi(x)=x\exp(-\alpha
   x^2)$ sea función propia del operador 
   $\hat{\Omega}=\frac{d^2}{dx^2} + bx^2$.
   Normalizar la función y calcular el valor 
   propio de $x$ para un
   sistema descrito por dicha función. Para resolver
   este ejercicio es necesaria la integral
   \begin{equation}
       \int_0^{\infty}x^{2n}e^{-ax^2}dx = 
       \frac{1\cdot 3\cdot 5... (2n-1)}{2^{n+1}a^n}\sqrt{\frac{\pi}{a}}
   \end{equation}
   
   \item Calcula el valor medio del momento y
   la posición para el caso de una partícula de
   masa $m$ en una caja unidimensional de anchura
   $L$. Comprueba que se cumple el principio de
   incertidumbre.
   
   \item Sea una partícula de masa $m$ que se
   mueve en una caja de potencial de anchura $L$.
   (a) Calcula la probabilidad de que la partícula
   se encuentre entre $L/4$ y $3L/4$. (b) Evalua el
   límite clásico.
   
  \item Un rodamiento de masa 1 g se encuentra
  en una caja unidimensional de anchura 10 cm
  moviéndose a una velocidad de 1 cm/s. (a) Calcula
  el número cuántico. (b) Muestra cuál es el especiamiento
  entre dos niveles para el valor del número cuántico 
  obtenido en el apartado anterior. (c) ¿Qué ilustran los 
  resultados?
  
  \item Una masa de 45 g en un muelle oscila a la 
  frecuencia de 2.4 vibraciones por segundo con 
  una amplitud de 4 cm. (a) Calcula la constante 
  de fuerza del muelle. (b) ¿Cuál sería el número
  cuántico $v$ si el sistema se tratase 
  mecanocuánticamente?
  
  \item Calcula la frecuencia de la radiación emitida
  cuando un oscilador armónico de frecuencia $6\times 10^{13}$
  s$^{-1}$ salta del nivel 8 al 7.
  
  \item Comprueba que el armónico esférico $Y_{10}$ es una solución 
  de la ecuación $\Lambda^2Y_{l,m_l}=-l(l+1)Y_{l,m_l}$.
\end{enumerate}

\newpage
\subsection{\textbf{Soluciones:}}
\begin{enumerate}
    \item (a) $E_{max}=5.56\times 10^{-20}$ J$=0.34$ eV. (b) $\lambda=554$ nm.
    \item $\lambda=3.32\times 10^{-34}$ m.
    \item $\Delta x=3.85\times 10^{-4}$m.
    \item (a) $\psi=(\frac{1}{\sqrt{2}})(\psi_1+\psi_2)$. (b) $\psi=(\frac{1}{\sqrt{2}})(\psi_1-\psi_2)$
    \item (a) No. (d) No. (c) Sí. (d) No. (e) Sí.
    \item (a) Tanto $\hat{x}$ como $\hat{p}_x$ son hermíticos. (b) No conmutan. (c) Ambos son lineales. (d) La función $\psi$ no es función propia ni de $\hat{x}$ ni de $\hat{p}_x$.
    \item La constante de normalización es $A=\sqrt{2/a}$ (ver partícula en una caja).
    \item La función $\psi$ es función propia si $b=-4\alpha^2$. La constante de normalización es
    $N=(\frac{2}{\sqrt{\pi/(2a)^3}})^{1/2}$. El valor
    promedio de la posición es $\langle x \rangle=0$ (la función que debemos integrar es una función impar).
    \item $\langle p_x\rangle =0$;  $\langle x\rangle =L/2$. El principio de incertidumbre
    se cumple (para verificarlo, hemos de probar que $\Delta x \Delta p_x\geq \hbar/2$; para lo que nos hace faltan $\langle x^2\rangle$ y $p_x^2$).
   \item $P=\frac{1}{2}-\frac{1}{2n\pi}(\sin \frac{3n\pi}{2} - \sin\frac{n\pi}{2})$. En el límite clásico: $\lim_{n\rightarrow \infty}P=1/2$.
   \item (a) $n=3\times 10^{27}$. (b) $\Delta E\simeq 3\times 10^{-35}$ J. (c) Principio de
   correspondencia.
   \item (a) $k=10.2$ N/m. (b) $v=5.16\times 10^{30}$.
  \item $\nu=6\times 10^{13}$s$^{-1}$.   
  \item El valor propio del operador es -2, que en
  efecto es igual a $-l(l+1)$ para $l=1$.
\end{enumerate}
\end{document}